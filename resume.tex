\documentclass[10pt]{article}
\usepackage{array, xcolor, graphicx, longtable}
\usepackage[margin=2.5cm]{geometry}
\setlength\parindent{0pt}

\title{\bfseries\Huge Bartosz Golaszewski}
\author{brgl@bgdev.pl}
\date{}

\definecolor{lightgray}{gray}{0.77}
\newcolumntype{L}{>{\raggedleft}p{0.17\textwidth}}
\newcolumntype{R}{p{0.785\textwidth}}
\newcommand\VRule{\color{lightgray}\vrule width 0.5pt}

\begin{document}

	\begin{minipage}{0.65\textwidth}
		\begingroup
		\let\center\flushleft
		\let\endcenter\endflushleft
		\maketitle
		\endgroup
	\end{minipage}
	\begin{minipage}{0.3\textwidth}
		\flushright{\includegraphics[width=3.5cm]{./pict.jpg}}
	\end{minipage}
	\vspace{2em}
	\linebreak
	\begin{minipage}[ht]{0.68\textwidth}
		Date and place of birth: June 10, 1987, Bydgoszcz, Poland\\
		Address: Piastowa 16/7, 85-790 Bydgoszcz, Poland\\
		Phone: +48 575 310 361
	\end{minipage}

	\vspace{20pt}

\section*{Summary}
Embedded linux expert and open source contributor.\\

Very good knowledge of the complete software stack - experience ranging from low-level, real-time
operating systems, through the linux kernel to user-space programs, libraries and distributions.\\

Experienced in interacting with hardware (board bring-up, writing device drivers, general kernel
space development).\\

Worked on international projects in a broad range of fields: bleeding edge consumer electronics,
high availability systems and military applications. Both on-site as well as remotely.\\

Significant contributions to many open-source projects - especially in the embedded applications.
Experienced in working with the open-source community.\\

Using C for low-level development, C++, python and bash/sh for high-level use cases.\\

Speaker at several international conferences.\\

Maintainer of libgpiod - C library and tools for accessing the GPIO character device on linux.\\

\section*{Professional Experience}
\begin{longtable}{L!{\VRule}R}
X 2014--III 2017&\textbf{BayLibre}\\&
\\&
Embedded Linux Engineer.\\&
\\&
\textbf{Highlights:}\\&
Upstream linux development for Texas Instruments DaVinci boards - worked on the \textbf{SATA},
\textbf{VPIF} and \textbf{LCD} controllers support.\\&
\\&
Upstream merging of BayLibre's \textbf{ACME} (Another Cute Measurement Equipment) \textbf{sigrok}
driver, sigrok buildroot packages and extensions to \textbf{at24}, \textbf{ina2xx} and
\textbf{tmp401} linux drivers.\\&
\\&
Contributions to the irq and \textbf{GPIO} subsystems in the linux kernel.\\&
\\&
Prepared the open-source software suite for BayLibre's ACME power-measurement cape for BeagleBone
Black. Presented the cape during the \textbf{Embedded Linux Conference Europe} 2014 in
Dusseldorf, Germany.\\&
\\
VI 2015--IX 2016&\textbf{Project Ara/Google ATAP}\\&
\\&
Firmware and Kernel Developer.\\&
\\&
\textbf{Highlights:}\\&
Significant contributions to the \textbf{module firmware}. Involved in power management, device
driver model, core system architecture and hardware support development. Minor contributions to the
related linux kernel code (now merged upstream).\\&
\\
IV 2014--IX 2014&\textbf{Intel Corporation}\\&
\\&
Embedded Linux \& Android Engineer\\&
\\&
\textbf{Highlights:}\\&
Implementation, support and bugfixing of \textbf{Linux device drivers} mostly for \textbf{WiFi}
and \textbf{bluetooth} devices used in Intel reference boards. Integration of Intel devices with
the Linux kernel and \textbf{Android framework}.\\&
\\
XI 2012--III 2014&\textbf{Amadeus IT Group}\\&
\\&
Software Developer in the Service Integrator team\\&
\\&
\textbf{Highlights:}\\&
Development and support of the \textbf{Service Integrator} project -- a fast, scalable message bus
and \textbf{IPC} system used throughout the Amadeus Airline Reservation System infrastructure --
using C++ and Python programming languages.\\&
\\
VIII 2009--X 2012&\textbf{TELDAT Sp. J. Kruszynski \& Cichocki}\\&
\\&
Embedded Software Engineer\\&
\\&
\textbf{Highlights:}\\&
Development and support of the linux-based operating system and its applications running on the
\textbf{military telecommunications} equipment manufactured by the company. Taking care of the
complete software stack: from board bring-up and linux device drivers to user space programs and
libraries.\\&
\end{longtable}

\section*{Education}
\begin{longtable}{L!{\VRule}R}
2009--2012&
Graduate studies at the \textbf{Nicolaus Copernicus University in Torun}, Faculty of Informatics,
thesis: \textbf{"Selected data transmission security issues in RPC protocols for embedded
systems."} - composed of a theoretical part and an example implementation of a simple and secure
RPC protocol for embedded, low-resource systems,\\[5pt]

X 2008--III 2009&
Georg-August-Universitaet in Goettingen (Germany) as a fellow of the ERASMUS programme,\\[5pt]

2006--2009&
undergraduate studies at the University of Nicolaus Copernicus University in Torun, Faculty of
Informatics,\\[5pt]

\end{longtable}

\section*{Open source activity}
\begin{itemize}
	\item significant contributions to the \textbf{linux kernel}: improved support for the
	Texas Instruments DaVinci da850 \textbf{LCD}, \textbf{SATA}, \textbf{VPIF} and
	\textbf{memory controllers}, improvements in \textbf{ina2xx} and \textbf{tmp401} drivers,
	new helper macros, clean-up of the cputopology code, support for
	\textbf{at24cs} and \textbf{at24mac} EEPROM series, extensions of the gpio-mockup driver
	for testing purposes,
	\item authored \textbf{several busybox applets} (i2c-tools, nsenter, unshare, shuf,
	unit testing framework) as well as submitted many bug-fixes and improvements,
	\item developer of several buildroot packages,
	\item one of the key contributors to \textbf{Project Ara module firmware} (device driver
	architecture, power-management, core-system, hardware support),
	\item author of the \textbf{baylibre-acme} driver for sigrok, implemented several new
	features and bug-fixes for the whole sigrok suite,
	\item minor contributions to uClibc, kmod \& u-boot,
	\item author and maintainer of \textbf{libgpiod} - a C library and tools for interacting
	with the linux GPIO character device
	\item smaller projects hosted on \textbf{github} -- username: brgl.
\end{itemize}

\section*{Conferences and presentations}
\begin{itemize}
	\item Presented BayLibre's ACME cape during the technical showcase at the Embedded Linux
	Conference Europe 2014 in Dusseldorf,
	\item delivered the presentation \textit{"Sigrok: Adventures in Integrating a
	Power-Measurement Device"} during the Embedded Linux Conference 2015 in San Jose, CA and,
	again, presented ACME during the technical showcase.
	\item delivered the presentation \textit{"Useful systemd functionalities without systemd"}
	during the Embedded Linux Conference Europe 2015 in Dublin, Ireland.
\end{itemize}

\section*{Languages}
\begin{tabular}{L!{\VRule}R}
Polish&native speaker (mother tongue),\\
English&full professional proficiency,\\
German&professional working proficiency (a total of four years of documented residence and
education in Germany),\\
French&professional working proficiency (four years of documented residence in France).\\
\end{tabular}

\end{document}
